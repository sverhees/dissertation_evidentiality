
\begin{appendices}
\chapter{Транскрипция} \label{app1}

\begin{table}[ht]
\caption{Соответствия транскрипций}
\vspace{0.3cm}
\centering
\label{tab:transcr}
\begin{tabular}{cccc}

МФА	&	\citep{tsakhurgram}	&	\citep{bagvalalgram}	&	Наст.	\\ \hline
i	&	i	&	i	&	i	\\
ɨ	&	ɨ	&		&	ɨ	\\
u	&	u	&	u	&	u	\\
e	&	e	&	e	&	e	\\
o	&	o	&	o	&	o	\\
ɑ	&	a	&	a	&	a	\\
ː	&	ˉ	&	ˉ	&	ː	\\
̃	&		&	̃	&	\textasciitilde{}	\\
m	&	m	&	m	&	m	\\
n	&	n	&	n	&	n	\\
l	&	l	&	l	&	l	\\
r	&	r	&	r	&	r	\\
w	&	w	&	w	&	w	\\
j	&	j	&	j	&	j	\\
ʔ	&	ʔ	&	ʔ	&	ʔ	\\
h	&	h	&	h	&	h	\\
ʕ	&		&	ʕ	&	ʕ	\\
ħ	&		&	H	&	ħ	\\
ˁ	&	I	&		&	ˁ	\\
b	&	b	&	b	&	b	\\
d	&	d	&	d	&	d	\\
	&	ǯ	&	ǯ	&	dž	\\
g	&	g	&	g	&	g	\\
ɢ	&	G	&		&	ɢ	\\
p	&	p	&	p	&	p	\\
t	&	t	&	t	&	t	\\
ts	&	c	&	c	&	c	\\
tʃ	&	č	&	č	&	č	\\
k	&	k	&	k	&	k	\\
q	&	q	&	q	&	q	\\
p’	&	p’	&		&	p’	\\
t’	&	t’	&	t’	&	t’	\\
tɬ’	&		&	L’	&	ƛ’	\\
ts’	&	c’	&	c’	&	c’	\\
tʃ’	&	č’	&	č’	&	č’	\\
k’	&	k’	&	k’	&	k’	\\
q’	&	q’	&	q’	&	q’	\\
z	&	z	&	z	&	z	\\
ʐ	&		&	ž	&	ž	\\
ɣ	&	ǧ	&		&	ɣ	\\
\end{tabular}
\end{table}

\pagebreak

\begin{table}[H]
\caption{Соответствия транскрипций (продолжение)}
\vspace{0.3cm}
\centering
\label{tab:transcr2}
\begin{tabular}{cccc}

МФА	&	\citep{tsakhurgram}	&	\citep{bagvalalgram}	&	Наст.	\\ \hline
ʁ	&	R	&	R	&	ʁ	\\
f	&	(f)	&		&	f	\\
ɬ	&		&	ł	&	ɬ	\\
s	&	s	&	s	&	s	\\
ʂ	&	š	&	š	&	š	\\
x	&	x	&	x	&	x	\\
χ	&	X	&	X	&	χ	\\
s’	&	s’	&		&	s’	\\
ʃ’	&	š’	&		&	š’	\\
ː	&	ˉ	&	ˉ	&	ː	\\
ʲ	&	\textsubscript{j, ɣ}	&		&	ʲ	\\
ʷ	&	˳	&	˳	&	ʷ	
\end{tabular}
\end{table}


\chapter{Элицитированные нарративы на андийском языке} \label{app2}

\section{Анкета}

\subsection{Косвенная засвидетельствованность}

(1.1) Мне рассказывали, что однажды моя бабушка шла по лесу, собирала ягоды. (1.2) Долго она ходила и собирала. (1.3) В какой-то момент она очень устала. (1.4) Она села на ствол дерева. (1.5) Как будто из ничего, появилась змея. (1.6) Бабушка испугалась. (1.7) Она (нечаянно) наступила на змею. (1.8) Змея укусила ее в ногу. (1.9) Она [бабушка] взяла камень, и бросила [его] на змею. (1.10) Та [змея] умерла. 

\subsection{Прямая засвидетельствованность}

(2.1) Несколько лет назад мы с братом шли по лесу и собирали ягоды. (2.2) Долго мы ходили и собирали. (2.3) В какой-то момент мы очень устали. (2.4) Мы сели на землю под деревом. (2.5) Как будто из ничего, появилась змея. (2.6) Мы испугались. (2.7) Брат нечаянно наступил на нее. (2.8) Змея укусила его в ногу. (2.9) Он взял камень, и бросил его на змею. (2.10) Змея умерла. 

\section{Переводы}

При элицитации мы попросили носителей перевести историю по предложениям устно: сначала первую (косвенную) версию, потом вторую. Ниже мы приводим отглоссированные переводы --- в качестве перевода приводится исходный стимул. Для каждого носителя указан код и село (носители при этом упорядочены по селу).

\subsection{NNA - Муни}

\lb{NNA11}{\gll di-qi bosono b-iʁi iši j-eč'uχa ila reš-du=lo j-onno rešu-ƛi piqi b-ak'aru-ma iʁi=no \\
{\First}{\Sg}-{\Instr} tell.{\Cvb} {\N}-be.{\Aor} {\First}.{\Excl} {\F}-big mother forest-{\Lat}={\Add} {\F}-go.{\Cvb} forest-{\Inter} fruit {\N}-gather-{\Hab} be={\Add} \\
\trans (1.1) Мне рассказывали, что однажды моя бабушка шла по лесу, собирала ягоды.}

\lb{NNA12}{\gll iha rihi-di=godi heti hewi b-ak'arunni \\
a\_lot\_of time-{\Erg}={\Rep} {\Dem} {\Dem} {\N}-gather.{\Aor} \\
\trans (1.2) Долго она ходила и собирала.}

\lb{NNA13}{\gll sewi zamanalla-ʔa he j-aʁodi\\
one time-{\Sup} {\Dem} {\F}-become\_tired.{\Aor}.{\Rep} \\
\trans (1.3) В какой-то момент она очень устала.}

\lb{NNA14}{\gll he hok'u (j)-iʁi-lo biχiwkala-ʔa\\
{\Dem} down ({\F})-be-{\Pf} ?-{\Sup}\\
\trans (1.4) Она села на ствол дерева. }

\lb{NNA15}{\gll eχːudu b-iʁi-ɬu b-iʁi-lo b-uχ-odi birka\\
after {\N}-be-? {\N}-be-{\Cvb} {\N}-appear-{\Aor}.{\Rep} snake\\
\trans (1.5) Как будто из ничего, появилась змея.}

\lb{NNA16}{\gll j-eč'uχa ila ƛ'irdi-lo\\
{\F}-big mother become\_frightened-{\Pf} \\
\trans (1.6) Бабушка испугалась.}

\lb{NNA17}{\gll heti c'inni-džigu birkala-ʔa č'ik'a b-iʁi-lo\\
{\Dem} know-{\Neg}.{\Cvb} snake-{\Sup} foot {\N}-be-{\Pf} \\
\trans (1.7) Она (нечаянно) наступила на змею.\footnote{\textit{c'inni-džigu} (know-{\Neg}.{\Cvb}) букв. `не зная' иногда переводится как `вдруг'.}}

\lb{NNA18}{\gll birkalo-di heƛi č'ik'a q'ami-lo\\
snake-{\Erg} {\Dem} foot bite-{\Pf} \\
\trans (1.8) Змея укусила ее в ногу.}

\lb{NNA19}{\gll heti hinc'o=lo b-iχi-lo birkala-ʔa gavi-lo\\
{\Dem} stone={\Add} {\N}-take-{\Pf} snake-{\Sup} throw-{\Pf} \\
\trans (1.9) Она [бабушка] взяла камень, и бросила [его] на змею.}

\lb{NNA110}{\gll birka b-iƛ'*o-lo\\
snake {\N}-die-{\Pf} \\
\trans (1.10) Та [змея] умерла.\footnote{Символ \textit{ƛ'*} обозначает смягченный вариант \textit{ƛ'}, который в мунинском диалекте противопоставлен \textit{ƛ'}.}}
 

\lb{NNA21}{\gll čongulo rešin siwaχadi den wocːi=lo taqi reš-tu w-oʔonno=ʁi rešu-ƛi piqi b-ak'arun-nu-ɬiri\\
some year ago {\First}{\Sg} brother={\Add} ? forest-{\Lat} {\M}-go.{\Cvb}=be.{\Aor} forest-{\Inter} fruit {\N}-gather-{\Inf}-{\Purp} \\
\trans (2.1) Несколько лет назад мы с братом шли по лесу и собирали ягоды.}

\lb{NNA22}{\gll iši-di hewi iha rihi-di b-ak'aruni\\
{\First}.{\Excl}-{\Erg} {\Dem} a\_lot\_of time-{\Erg} {\N}-gather.{\Aor}\\
\trans (2.2) Долго мы ходили и собирали.}

\lb{NNA23}{\gll se(b) zamanalla-ʔa išili t'uruħala w-aʁi\\
one({\N}) time-{\Sup} {\First}.{\Excl} very {\M}-become\_tired.{\Aor} \\
\trans (2.3) В какой-то момент мы очень устали.}

\lb{NNA24}{\gll išili rišaɬulo-č'u raƛ'ila-ʔa hok'u uʁi\\
{\First}{\Excl} tree-{\Cont} ground-{\Sup} down be.{\Aor}\\
\trans (2.4) Мы сели на землю под деревом. }

\lb{NNA25}{\gll b-iʁi-ɬu b-iʁi-lo b-uχːi berka\\
{\N}-be-? {\N}-be-{\Cvb} {\N}-appear.{\Aor} snake\\
\trans (2.5) Как будто из ничего, появилась змея.}

\lb{NNA26}{\gll išili ƛ'irdi \\
{\First}.{\Excl} become\_frightened.{\Aor} \\
\trans (2.6) Мы испугались.}

\lb{NNA27}{\gll c'inni-džigu wocːiš-di birkala-ʔa č'ik'a b-iʁi\\
know-{\Neg}.{\Cvb} brother-{\Erg} snake-{\Sup} foot {\N}-be.{\Aor}\\
\trans (2.7) Брат нечаянно наступил на нее.\footnote{\textit{c'inni-džigu} (know-{\Neg}.{\Cvb}) букв. `не зная' иногда переводится как `вдруг'.}}

\lb{NNA28}{\gll birkalo-di hešuwi č'ik'a q'ammi\\
snake-{\Erg} {\Dem} foot bite.{\Aor} \\
\trans (2.8) Змея укусила его в ногу. }

\lb{NNA29}{\gll heš-di hinc'o=lo biχi-lo birkalaʔa gawi\\
{\Dem}-{\Erg} stone={\Add} take-{\Cvb} snake-{\Sup}.{\Lat} throw.{\Aor} \\
(2.9) Он взял камень, и бросил его на змею. }

\lb{NNA210}{\gll birka b-iƛ'*o\\
snake {\N}-die.{\Aor} \\
(2.10) Змея умерла.\footnote{Символ \textit{ƛ'*} обозначает смягченный вариант \textit{ƛ'}, который в мунинском диалекте противопоставлен \textit{ƛ'}.}}

\subsection{A - Риквани}

\lb{A11}{\gll di-ɬu bosw-ado b-ik’o se-b zaman di-j j-eč’uχa baba reš-ƛi hirʁalol r-ak’ar-ado j-ik’o\\
{\First}{\Sg}-{\Dat} tell-{\Prog} {\Inanone}-be.{\Aor} one-{\Inanone} time {First}{\Sg}-{\F}[{\Gen}] {\F}-big mom forest-{Inter} raspberry.{\Pl} {\Inantwo}-gather-{\Prog} {\F}-be.{\Aor}\\
\trans (1.1) Мне рассказывали, что однажды моя бабушка шла по лесу, собирала ягоды.}

\lb{A12}{\gll b-ihu zaman j-ik’o hege-j r-ak’ar-ado\\
{\Inanone}-a\_lot\_of time {\F}-be.{\Aor} {\Dem}-{\F} {\Inantwo}-gather-{\Prog}\\
\trans (1.2) Долго она ходила и собирала. }

\lb{A13}{\gll se-b zaman hege-j zolo j-aʁi\\
one-{\Inanone} time {\Dem}-{\F} very {\F}-become\_tired.{\Aor}\\
\trans (1.3) В какой-то момент она очень устала.}

\lb{A14}{\gll i hoɢu j-ik’o kuc’odo-l’a\\
and down {\F}-be.{\Aor} treetrunk.{\Sup}\\
\trans (1.4) Она села на ствол дерева. }

\lb{A15}{\gll (sebgulol-ɬu sːu-b-ɬu-kːu) b-uχːi-d berka\\
(nothing-{\Adv} {\Neg}.{\Cop}-{\Pst}.{\Ptcp}-{\Adv}-{\El}) {\Inanone}-appear-{\Pf} snake\\
\trans (1.5) Как будто из ничего, появилась змея. }

\lb{A16}{\gll jaja siri\\
mom become\_frightened.{\Aor}\\
\trans (1.6) Бабушка испугалась.}

\lb{A17}{\gll c’inni<gu>-čigu (hegel-d) berku-l’a č’ek’a b-iʁ-oɬi\\
know<{\Emph}>-{\Neg}.{\Cvb} ({\Dem}-{\Erg}) snake-{\Sup} foot {\Inanone}-stop-{\Caus}.{\Aor}\\
\trans (1.7) Она (нечаянно) наступила на змею.}

\lb{A18}{\gll berku-d hegel-ƛ č’ek’a k’ammi-d\\
snake-{\Erg} {\Dem}-{\Gen} foot bite-{\Pf}\\
\trans (1.8) Змея укусила ее в ногу.}

\lb{A19}{\gll hegel-d b-iχi-d=lo kodi hinc’o šammi hegel-lo (berku-lo) hinc’o\\
{\Dem}-{\Erg} {\Inanone}-take-{\Cvb}={\Add} in\_hands stone throw.{\Aor} {\Dem}-{\Sup}.{\Lat} (snake-{\Sup}.{\Lat}) stone\\
\trans (1.9) Она [бабушка] взяла камень, и бросила [его] на змею.}

\lb{A110}{\gll berka b-ič’o-d\\
snake {\An}-die-{\Pf}\\
\trans (1.10) Та [змея] умерла.}
 
\lb{A21}{\gll čomlo rešin sedu den wocː=logu reš-ƛi hirʁalol r-ak’ar-ado w-ok’o\\
some year ago {\First}{\Sg} brother={\Com} forest-{\Inter} raspberry.{\Pl} {\Inantwo}-gather-{\Prog} {\M}-{\Pl}.be.{\Aor}\\
\trans (2.1) Несколько лет назад мы с братом шли по лесу и собирали ягоды.}

\lb{A22}{\gll b-ihu zaman w-ok’o išːil r-ak’ar-ado\\
{\Inanone}-a\_lot\_of time {\M}-{\Pl}.be.{\Aor} {\First}.{\Excl} {\Inantwo}-gather-{\Prog}\\
\trans (2.2) Долго мы ходили и собирали.}

\lb{A23}{\gll se-n zaman-na-kːu išːil zolo w-aʁi\\
one-{\Inanone} time-{\Sup}-{\El} {\First}.{\Excl} {\M}-become\_tired.{\Aor}\\
\trans (2.3) В какой-то момент мы очень устали. }

\lb{A24}{\gll išːil hoɢu w-ok’o ƛ’eturu-ƛ’ hiƛ’u\\
{\First}.{\Excl} down {\M}-{\Pl}.be.{\Aor} tree-{\Sub} under\\
\trans (2.4) Мы сели на землю под деревом. }

\lb{A25}{\gll (sebgulol-ɬu sːu-b-ɬu-kːu) b-uχːi-d b-iɢo berka\\
(nothing-{\Adv} {\Neg}.{\Cop}-{\Pst}.{\Ptcp}-{\Adv}-{\El}) {\Inanone}-appear-{\Cvb} {\Inanone}-come.{\Aor} snake\\
\trans (2.5) Как будто из ничего, появилась змея.}

\lb{A26}{\gll išːil siri\\
{\First}.{\Excl} become\_frightened.{\Aor}\\
\trans (2.6) Мы испугались.}

\lb{A27}{\gll wocːu-d c’inni<gu>-č’igu hegel-l’a ček’a b-iʁ-oɬi\\
brother-{\Erg} know<{\Emph}>-{\Neg}.{\Cvb} {\Dem}-{\Sup} foot {\Inanone}-stop-{\Caus}.{\Aor}\\
\trans (2.7) Брат нечаянно наступил на нее.}

\lb{A28}{\gll berku-d hegešːu-b č’ek’a k’ammi\\
snake-{\Erg} {\Dem}-{\Inanone}[{\Gen}] foot bite.{\Aor}\\
\trans (2.8) Змея укусила его в ногу. }

\lb{A29}{\gll hegešːu-d kodi=lo b-iχi-d hinc’o hegel-l’a džaj\\
{\Dem}-{\Erg} in\_hands={\Add} {\N}-take-{\Cvb} stone {\Dem}-{\Sup} hit.{\Aor}\\
\trans (2.9) Он взял камень, и бросил его на змею. }

\lb{A210}{\gll onšlo berka b-ič’o\\
then snake {\An}-die.{\Aor}\\
\trans (2.10) Змея умерла.}

\subsection{ABE - Риквани}

\lb{ABE11}{\gll di-ɬu bosːon se-r zuw di-j j-eč’uχa baba j-il’in-no j-ik’o reš-ƛi-kːu, riχilol r-ak’ar-ado=guža\\
{\First}{\Sg}-{\Dat} tell.{\Aor} one-{\Inantwo} day {\First}{\Sg}-{\F}[{\Gen}] {\F}-big mom {\F}-go-{\Hab} {\F}-be.{\Aor} forest-{\Inter}-{\El} strawberry.{\Pl} {\Inantwo}-gather-{\Prog}={\Sim}\\
\trans (1.1) Мне рассказывали, что однажды моя бабушка шла по лесу, собирала ягоды.}

\lb{ABE12}{\gll hege-j b-ihu=gu zaman j-il’in-no j-ik’o b-ak’ar-ado=guža\\
{\Dem}-{\F} {\Inanone}-a\_lot\_of={\Emph} time {\F}-go-{\Hab} {\F}-be.{\Aor} {\Inanone}-gather-{\Prog}={\Sim}\\
\trans (1.2) Долго она ходила и собирала. }

\lb{ABE13}{\gll se-b=ɢa zaman hege-j zolo j-aʁi\\
one-{\Inanone}={\Indef} time {\Dem}-{\F} very {\F}-become\_tired{\Aor}\\
\trans (1.3) В какой-то момент она очень устала.}

\lb{ABE14}{\gll hege-j hoɢu j-ik’o rešu-l’a\\
{\Dem}-{\F} down {\F}-be.{\Aor} tree-{\Sup}\\
\trans (1.4) Она села на ствол дерева. }

\lb{ABE15}{\gll se-b=ɢa zaman sebgulo sːu-b-ɬu-kːu b-uχːi berka\\
one-{\Inanone}={\Indef} time nothing {\Neg}.{\Cop}-{\Pst}.{\Ptcp}-{\Adv}-{\El} {\An}-appear snake\\
\trans (1.5) Как будто из ничего, появилась змея. }

\lb{ABE16}{\gll j-eč’uχa baba siri=ɬodi\\
{\F}-big mom become\_frightened.{\Aor}={\Rep}\\
\trans (1.6) Бабушка испугалась.}

\lb{ABE17}{\gll hege-j c’inni<gu>-čigu berku-l’a č’ek’a b-iʁ-oɬi\\
{\Dem}-{\F} know<{\Emph}>-{\Neg}.{\Cvb} snake-{\Sup} foot {\Inanone}-stop-{\Caus}.{\Aor}\\
\trans (1.7) Она (нечаянно) наступила на змею.}

\lb{ABE18}{\gll berku-d hege-l q’ammi=ɬodi č’ek’u-č’u\\
snake-{\Erg} {\Dem}-? bite.{\Aor}={\Rep} foot-{\Cont}\\
\trans (1.8) Змея укусила ее в ногу.}

\lb{ABE19}{\gll j-eč’uχa.baba-d hinc’o b-iχi-d=lo džaj berku-l’a\\
{\F}big.mom-{\Erg} stone {\Inanone}-take-{\Cvb}={\Add} hit.{\Aor} snake-{\Sup}\\
\trans (1.9) Она [бабушка] взяла камень, и бросила [его] на змею.}

\lb{ABE110}{\gll hege-b b-ič’o=ɬodi\\
{\Dem}-{\An} {\An}-die.{\Aor}={\Rep}\\
\trans (1.10) Та [змея] умерла.}

\lb{ABE21}{\gll čom rešin seda den=no di-w wocː=lo w-ol’in-no w-ok’o reš-ƛi-kːu, reχːilol r-ak’ar-ado=guža\\
some year ago {\First}{\Sg}={\Add} {\First}{\Sg}-{\M}[{\Gen}] brother={\Add} {\M}-{\Pl}.go-{\Hab} {\M}-{\Pl}.be.{\Aor} forest-{\Inter}-{\El} strawberry.{\Pl} {\Inantwo}-gather-{\Prog}={\Sim}\\
\trans (2.1) Несколько лет назад мы с братом шли по лесу и собирали ягоды.}

\lb{ABE22}{\gll b-ihu zaman w-oƛo išːil r-ak’ar-ado=guža\\
{\Inanone}-a\_lot\_of time {\M}-{\Pl}.walk.{\Aor} {\First}.{\Excl} {\Inantwo}-gather-{\Prog}={\Sim}\\
\trans (2.2) Долго мы ходили и собирали.}

\lb{ABE23}{\gll se-b=ɢa zaman išːil zolo w-aʁi\\
one-{\Inanone}={\Indef} time {\First}.{\Excl} very {\M}-become\_tired.{\Aor}\\
\trans (2.3) В какой-то момент мы очень устали. }

\lb{ABE24}{\gll išːil ƛ’et’uru-ƛ’ hiƛ’u hoɢu w-ok’o\\
{\First}.{\Excl} tree-{\Sub} under down {\M}-{\Pl}.be{\Aor}\\
\trans (2.4) Мы сели на землю под деревом. }
 
\lb{ABE25}{\gll se-b=ɢa zaman c’inni<gu>-č’igu b-uχːi berka\\
one-{\Inanone}={\Indef} time know<{\Emph}>-{\Neg}.{\Cvb} {\An}-appear.{\Aor} snake\\
\trans (2.5) Как будто из ничего, появилась змея.}

\lb{ABE26}{\gll išːil zolo siri\\
{\First}.{\Excl} very become\_frightened.{\Aor}\\
\trans (2.6) Мы испугались.}

 \lb{ABE27}{\gll di-w wocː c’inni<gu>-č’igu hegel-l’a č’ek’a b-iʁ-oɬi\\
{\First}{\Sg}-{\M}[{\Gen}] brother know<{\Emph}>-{\Neg}.{\Cvb} {\Dem}-{\Sup} foot {\Inanone}-stop-{\Caus}.{\Aor}\\
\trans (2.7) Брат нечаянно наступил на нее.}

\lb{ABE28}{\gll berku-d q’ammi hegešːu-b č’ek’a\\
snake-{\Erg} bite.{\Aor} {\Dem}-{\Inanone}[{\Gen}] foot \\
\trans (2.8) Змея укусила его в ногу.}

\lb{ABE29}{\gll hegešːu-d b-iχi hinc’o berku-l’a džaj\\
{\Dem}-{\Erg} {\Inanone}-take.{\Aor} snake-{\Sup} hit.{\Aor}\\
\trans (2.9) Он взял камень, и бросил его на змею. }

\lb{ABE210}{\gll berka b-ič’o\\
snake {\An}-die.{\Aor}\\
\trans (2.10) Змея умерла.}


\subsection{GRG - Риквани} 

\lb{GRG11}{\gll di-ɬu bosːon se-b miq’i-la di-j j-eč’uχa ila reš-ƛi-kːu j-il’in-no j-ik’o, hegel-d hurq’i r-ak’ar-ado r-ik’o\\
{\First}{\Sg}-{\Dat} tell.{\Aor} one-{\Inanone} road-{\Sup} {\First}{\Sg}-{\F}[{\Gen}] {\F}-big mother forest-{\Inter}-{\El} {\F}-go-{\Hab} {\F}-be.{\Aor} {\Dem}-{\Erg} strawberry {\Inantwo}-gather-{\Prog} {\Inantwo}-be.{\Aor}\\
\trans (1.1) Мне рассказывали, что однажды моя бабушка шла по лесу, собирала ягоды.}

\lb{GRG12}{\gll hegel-d b-ihu zaman b-uχi reš-ƛi hurq’i r-ak’ar-ado\\
{\Dem}-{\Erg} {\Inanone}-a\_lot\_of time {\Inanone}-disappear.{\Aor} forest-{\Inter} strawberry {\Inantwo}-gather-{\Prog}\\
\trans (1.2) Долго она ходила и собирала. }

\lb{GRG13}{\gll se-b zaman hege-j j-aʁi\\
one-{\Inanone} time {\Dem}-{\F} {\F}-become\_tired.{\Aor}\\
\trans (1.3) В какой-то момент она очень устала.}

\lb{GRG14}{\gll hege-j šan-nu-ri b-ukːu-b ƛ’et’uru-l’a hoɢu j-ik’o\\
{\Dem}-{\F} rest-{\Inf}-{\Purp} {\Inanone}-fall-{\Pst}.{\Ptcp} tree-{\Sup} down {\F}-be.{\Aor}\\
\trans (1.4) Она села на ствол дерева. }

\lb{GRG15}{\gll rok’ol-l’a=gu tigu b-uχːi berka\\
heart-{\Sup}={\Emph} suddenly {\An}-appear.{\Aor} snake\\
\trans (1.5) Как будто из ничего, появилась змея. }

\lb{GRG16}{\gll hege-j siri\\
{\Dem}-{\F} become\_frightened.{\Aor}\\
\trans (1.6) Бабушка испугалась.}

\lb{GRG17}{\gll hegel-d berku-l’a č’ek’a b-iʁ-oɬi-d\\
{\Dem}-{\Erg} snake-{\Sup} foot {\Inanone}-stop-{\Caus}.{\Aor}-{\Pf}\\
\trans (1.7) Она (нечаянно) наступила на змею.}

\lb{GRG18}{\gll berku-d hegel-ƛ č’ek’a q’ammi\\
snake-{\Erg} {\Dem}-{\Gen} foot bite.{\Aor}\\
\trans (1.8) Змея укусила ее в ногу.}

\lb{GRG19}{\gll hegel-d r-iχi-d=lo k’ant’a berku-l’a džaj\\
{\Dem}-{\Erg} {\Inantwo}-take-{\Cvb}={\Add} stick snake-{\Sup} hit.{\Aor}\\
\trans (1.9) Она [бабушка] взяла камень, и бросила [его] на змею.}

\lb{GRG110}{\gll berka b-ič’o\\
snake {\An}-die.{\Aor}\\
\trans (1.10) Та [змея] умерла.}
 
\lb{GRG21}{\gll čomlo rešin sedu=gu den=no wocːu-d=lo reš-ƛi hurq’i r-ak’ar-ado r-ik’o\\
some year ago={\Emph} {\First}{\Sg}.{\Erg}={\Add} brother-{\Erg}={\Add} forest-{\Inter} strawberry {\Inantwo}-gather-{\Prog} {\Inantwo}-be.{\Aor}\\
\trans (2.1) Несколько лет назад мы с братом шли по лесу и собирали ягоды.}

\lb{GRG22}{\gll išːil woχːuloq hurq’i r-ak’ar-ado w-ok’o\\
{\First}.{\Excl} ? strawberry {\Inantwo}-gather-{\Prog} {\M}-{\Pl}.be.{\Aor}\\
\trans (2.2) Долго мы ходили и собирали.}

\lb{GRG23}{\gll išːil zolo w-aʁi\\
{\First}.{\Excl} very {\M}-become\_tired.{\Aor}\\
\trans (2.3) В какой-то момент мы очень устали. }

\lb{GRG24}{\gll išːil ƛ’et’uru-ƛ’ hoɢu w-ok’o\\
{\First}.{\Excl} tree-{\Sub} down {\M}-{\Pl}.be.{\Aor}\\
\trans (2.4) Мы сели на землю под деревом.}

\lb{GRG25}{\gll rok’u-la=gu t’igu b-uχːi-d b-iɢo berka\\
heart-{\In}={\Emph} suddenly {\An}-appear-{\Cvb} {\An}-come.{\Aor} snake\\
\trans (2.5) Как будто из ничего, появилась змея.}

\lb{GRG26}{\gll išːil siri\\
{\First}.{\Excl} become\_frightened.{\Aor}\\
\trans (2.6) Мы испугались.}

\lb{GRG27}{\gll wocːu-d berku-l’a č’ek’a b-iʁ-oɬi\\
brother-{\Erg} snake-{\Sup} foot {\Inanone}-stop-{\Caus}.{\Aor}\\
\trans (2.7) Брат нечаянно наступил на нее.}

\lb{GRG28}{\gll berku-d hegešːu-b č’ek’a q’ammi\\
snake-{\Erg} {\Dem}-{\Inanone}[{\Gen}] foot bite.{\Aor}\\
\trans (2.8) Змея укусила его в ногу. }

\lb{GRG29}{\gll wocːu-d r-iχi k’ant’a=lojd hege-r berku-l’a džaj\\
brother-{\Erg} {\Inantwo}-take.{\Aor} stick={\Sbr} {\Dem}-{\Inantwo} snake-{\Sup} hit.{\Aor}\\
\trans (2.9) Он взял камень, и бросил его на змею. }

\lb{GRG210}{\gll berka b-ič’o\\
snake {\An}-die.{\Aor}\\
\trans (2.10) Змея умерла.}



\subsection{GRSh - Риквани}

\lb{GRSh11}{\gll di-ɬu bosːon di-j j-eč’uχa.baba j-uʔon=ɬoʁo reš-ƛi b-ak’arunni=ɬoʁo piq\\
{\First}{\Sg}-{\Dat} tell.{\Aor} {\First}{\Sg}-{\F}[{\Gen}] {\F}-big.mom {\F}-go.{\Aor}={\Quot} {\Inanone}-gather.{\Aor}={\Quot} fruit\\
\trans (1.1) Мне рассказывали, что однажды моя бабушка шла по лесу, собирала ягоды.}

\lb{GRSh12}{\gll hege-j b-ihu=gu=ri j-iʔin-no=lo j-ik’o-d b-ak’ar-ado b-ik’o-d\\
{\Dem}-{\F} {\Inanone}-a\_lot\_of={\Emph}=time {\F}-go-{\Hab}={\Add} {\F}-be-{\Cvb} {\Inanone}-gather-{\Prog} {\Inanone}-gather-{\Prog} {\Inanone}-be-{\Pf}\\
\trans (1.2) Долго она ходила и собирала. }

\lb{GRSh13}{\gll se-b zaman hege-j t’ulu=gu j-aʁi-d\\
one-{\N} time {\Dem}-{\F} very={\Emph} {\F}-become\_tired-{\Pf}\\
\trans (1.3) В какой-то момент она очень устала.}

\lb{GRSh14}{\gll hege-j hoɢ.ik’o-d ƛ’et’uro-ƛ angu-la\\
{\Dem}-{\F} down.be-{\Pf} tree-{\Gen} branch-{\Sup}\\
\trans (1.4) Она села на ствол дерева. }

\lb{GRSh15}{\gll c’inni<gu>-č’igu b-uχːi-d berka\\
know<{\Emph}>-{\Neg}.{\Cvb} {\An}-appear-{\Pf} snake\\
\trans (1.5) Как будто из ничего, появилась змея. }

\lb{GRSh16}{\gll j-eč’uχa.ba(ba) siri-d\\
{\F}-big.mom become\_frightened-{\Pf}\\
\trans (1.6) Бабушка испугалась.}

\lb{GRSh17}{\gll hege-j c’inni<gu>-č’igu b-iʁ-oɬi-d č’ek’a berku-la\\
{\Dem}-{\F} know<{\Emph}>-{\Neg}.{\Cvb} {\Inanone}-stop-{\Caus}.{\Aor} foot snake-{\Sup}\\
\trans (1.7) Она (нечаянно) наступила на змею.}

\lb{GRSh18}{\gll berku-d q’ammi-d hegel-č’u č’ek’u-ƛ\\
snake-{\Erg} bite-{\Pf} {\Dem}-{\Cont} foot-{\Gen}\\
\trans (1.8) Змея укусила ее в ногу.}

\lb{GRSh19}{\gll hegel-d b-iχi-d hinc’o onšːlo šammi-d berku-lo\\
{\Dem}-{\Erg} {\Inanone}-take-{\Pf} stone then throw-{\Pf} snake-{\Sup}.{\Lat}\\
\trans (1.9) Она [бабушка] взяла камень, и бросила [его] на змею.}

\lb{GRSh110}{\gll berka b-ič’o-(d)\\
snake {\An}-die.{\Aor}-({\Pf})\\
\trans (1.10) Та [змея] умерла.}
 
\lb{GRSh21}{\gll čom-bolo rešin sedu den=no di-w wocː=lo w-oʔon reš-ƛi b-ak’arunni piq\\
some-{\Indef} year ago {\First}.{\Sg}={\Add} {\First}{\Sg}-{\M}[{\Gen}] brother={\Add} {\M}-{\Pl}.go.{\Aor} forest-{\Inter} {\Inanone}-gather.{\Aor} fruit\\
\trans (2.1) Несколько лет назад мы с братом шли по лесу и собирали ягоды.}

\lb{GRSh22}{\gll išːil b-ihu=gu=ri w-oʔon-d=lo w-ok’o b-ak’ar-ado\\
{\First}.{\Excl} {\Inanone}-a\_lot\_of={\Emph}=time {\M}-{\Pl}.go-{\Cvb}={\Add} {\M}-{\Pl}.be.{\Aor} {\Inanone}-gather-{\Prog}\\
\trans (2.2) Долго мы ходили и собирали.}

\lb{GRSh23}{\gll se-b zaman išːil t’ulol=gu w-aʁi\\
one-{\Inanone} time {\First}.{\Excl} very={\Emph} {\M}-become\_tired.{\Aor}\\
\trans (2.3) В какой-то момент мы очень устали. }

\lb{GRSh24}{\gll išːil hoɢu w-ok’o ƛ’et’uru-ƛ’\\
{\First}.{\Excl} down {\M}-{\Pl}.be.{\Aor} tree-{\Sub}\\
\trans (2.4) Мы сели на землю под деревом. }

\lb{GRSh25}{\gll c’inni<gu>-č’igu inu-kːu-bolo b-uχːi berka\\
know<{\Emph}>-{\Neg}.{\Cvb} where-{\El}-{\Indef} {\An}-appear.{\Aor} snake\\
\trans (2.5) Как будто из ничего, появилась змея.}

\lb{GRSh26}{\gll išːil siri\\
{\First}.{\Excl} become\_frightened.{\Aor}\\
\trans (2.6) Мы испугались.}

\lb{GRSh27}{\gll wocː c’inni<gu>-č’igu b-iʁi č’ek’a berku-la\\
brother know<{\Emph}>-{\Neg}.{\Cvb} {\Inanone}-stop.{\Aor} foot snake-{\Sup}\\
\trans (2.7) Брат нечаянно наступил на нее.}

\lb{GRSh28}{\gll hegešː-č’u berku-d q’ammi-d č’ek’u-q\\
{\Dem}-{\Cont} snake-{\Erg} bite-{\Pf} foot-{\Instr}\\
\trans (2.8) Змея укусила его в ногу. }

\lb{GRSh29}{\gll hegešːu-d b-iχi-d=lo hinc’o šammi-d berku-lo\\
{\Dem}-{\Erg} {\Inanone}-take-{\Cvb}={\Add} stone throw-{\Pf} snake-{\Sup}.{\Lat}\\
\trans (2.9) Он взял камень, и бросил его на змею. }

\lb{GRSh210}{\gll berka b-ič’o\\
snake {\An}-die.{\Aor}\\
\trans (2.10) Змея умерла.}

\subsection{MShM - Риквани}

\lb{MShM11}{\gll di-ɬu bosːon onšː=gu j-oχoru-d rešli-l’o j-il’on-d r-ak’ar-ado r-ok’o riχilol \\
{\First}{\Sg}-{\Dat} tell.{\Aor} once={\Emph} {\F}-old-{\Erg} forest-{\In}.{\Lat} {\F}-go-{\Cvb} {\Inantwo}-gather-{\Prog} {\Inantwo}-{\Pl}.be.{\Aor} red\_berry.{\Pl}\\
\trans (1.1) Мне рассказывали, что однажды моя бабушка шла по лесу, собирала ягоды.}

\lb{MShM12}{\gll b-ihu zaman sori-d=lo r-ak’arunni heger-ul\\
{\Inanone}-a\_lot\_of time turn-{\Cvb}={\Add} {\Inantwo}-gather.{\Aor} {\Dem}.{\Inantwo}-{\Pl}\\
\trans (1.2) Долго она ходила и собирала. }

\lb{MShM13}{\gll se-b rihi hege-j j-aʁi-d=lo hoɢu j-ik’o šːan-nu-ri\\
one-{\Inanone} time {\Dem}-{\F} {\F}-become\_tired-{\Cvb}={\Add} down {\F}-be.{\Aor} rest-{\Inf}-{\Purp}\\
\trans (1.3) В какой-то момент она очень устала.}

\lb{MShM14}{\gll hoɢu j-ik’o ƛ’et’ru-ƛ angu-l’a\\
down {\F}-be.{\Aor} tree-{\Gen} branch-{\Sup}\\
\trans (1.4) Она села на ствол дерева. }

\lb{MShM15}{\gll se-b rihi inu-kkwollo b-uχːi-d b-iɢo berka\\
one-{\Inanone} time where-{\El}.{\Indef} {\An}-appear-{\Cvb} {\An}-come.{\Aor} snake\\
\trans (1.5) Как будто из ничего, появилась змея. }

\lb{MShM16}{\gll j-oχor siri\\
{\F}-old become\_frightened.{\Aor}\\
\trans (1.6) Бабушка испугалась.}

\lb{MShM17}{\gll hege-j j-iχːi-d=lo bužu-ro-l’a=gu č’ek’a b-iʁi berku-l’a\\
{\Dem}-{\F} {\F}-become\_confused-{\Cvb}={\Add} believe-{\Msd}-{\Sup} foot {\Inanone}-stop.{\Aor} snake-{\Sup}\\
\trans (1.7) Она (нечаянно) наступила на змею.}

\lb{MShM18}{\gll he-b=rihi berku-d hegel-č’u q’ammi\\
{\Dem}-{\Inanone}=time snake-{\Erg} {\Dem}-{\Cont} bite.{\Aor}\\
\trans (1.8) Змея укусила ее в ногу.}

\lb{MShM19}{\gll hegel-d b-iχi-d=lo hinc’o džaj berku-l’a\\
{\Dem}-{\Erg} {\Inanone}-take-{\Cvb}={\Add} stone hit.{\Aor} snake-{\Sup}\\
\trans (1.9) Она [бабушка] взяла камень, и бросила [его] на змею.}

\lb{MShM110}{\gll (he-rbihi) berka b-ič’o\\
({\Dem}-{\Inanone}.time) snake {\An}-die.{\Aor}\\
\trans (1.10) Та [змея] умерла.}
 
\lb{MShM21}{\gll se-r čom=lo rešin sedu den wocː=logu w-ul’on w-uk’o rešli ʁurq’i r-ak’arun-nu\\
one-{\Inantwo} some={\Add} year ago {\First}.{\Sg} brother={\Com} {\M}-go.{\Aor} {\M}-be.{\Aor} forest(?) wild\_strawberry {\Inantwo}-gather-{\Inf}\\
\trans (2.1) Несколько лет назад мы с братом шли по лесу и собирали ягоды.}

\lb{MShM22}{\gll b-ihu zaman išːi-d sori-d=lo qali ʁurq’i\\
{\Inanone}-a\_lot\_of time {\First}.{\Excl}-{\Erg} turn-{\Cvb}={\Add} search.{\Aor} wild\_strawberry\\
\trans (2.2) Долго мы ходили и собирали.}

\lb{MShM23}{\gll se-b zaman išːil w-aʁi\\
one-{\Inanone} time {\First}.{\Excl} {\M}-become\_tired.{\Aor}\\
\trans (2.3) В какой-то момент мы очень устали. }

\lb{MShM24}{\gll onšlo išːil hoɢu w-ok’o ƛ’et’uru-ƛ hiƛ’u hoɢol-l’a\\
then {\First}.{\Excl} down {\M}-{\Pl}.be.{\Aor} tree-{\Gen} under ground-{\Sup}\\
\trans (2.4) Мы сели на землю под деревом. }

\lb{MShM25}{\gll he-b zaman inu-kkwolo b-iɢo b-uχːi-d berka\\
{\Dem}-{\Inanone} time where-{\El}.{\Indef} {\An}-come.{\Aor} {\An}-appear-{\Cvb} snake\\
\trans (2.5) Как будто из ничего, появилась змея.}

\lb{MShM26}{\gll išːil siri\\
{\First}.{\Excl} become\_frightened.{\Aor}\\
\trans (2.6) Мы испугались.}

\lb{MShM27}{\gll wocːu-b č’ek’a b-iʁi bužu-rol-l’a=gu berku-l’a\\
brother-{\Inanone}[{\Gen}] foot {\Inanone}-stop.{\Aor} believe-{\Msd}-{\Sup}={\Emph} snake-{\Sup}\\
\trans (2.7) Брат нечаянно наступил на нее.}

\lb{MShM28}{\gll he-b=rihi berku-d q’ammi wocːu-b č’ek’u-č’u\\
{\Dem}-{\Inanone}=time snake-{\Erg} bite.{\Aor} brother-{\Inanone}[{\Gen}]-{\Cont}\\
\trans (2.8) Змея укусила его в ногу. }

\lb{MShM29}{\gll wocːu-d b-iχi-d=lo šammi hinc’o berku-l’o\\
brother-{\Erg} {\Inanone}-take-{\Cvb}={\Add} throw.{\Aor} stone snake-{\Sup}.{\Lat}\\
\trans (2.9) Он взял камень, и бросил его на змею. }

\lb{MShM210}{\gll (he-b=rihi) berka b-ič’o\\
({\Dem}-{\Inanone}=time) snake {\Inanone}-die.{\Aor}\\
\trans (2.10) Змея умерла.}



\subsection{AMKh - Рушуха} 

\lb{AMKh11}{\gll se-b onši iši-j ila reš-ƛi j-iʔon di-qi bosːon di-j ilu-di b-ak’arun c’orol.č’at’i\\
one-{\Inanone} time {\First}.{\Excl}-{\F}[{\Gen}] mother forest-{\Inter} {\F}-go.{\Aor} {\First}{\Sg}-{\Instr} tell.{\Aor} {\First}{\Sg}-{\F}[{\Gen}] mother-{\Erg} {\Inanone}-gather.{\Aor} wild\_raspberry\\
\trans (1.1) Мне рассказывали, что однажды моя бабушка шла по лесу, собирала ягоды.}

\lb{AMKh12}{\gll b-ihu=rij j-eƛi-du b-ak’arun\\
{\Inanone}-a\_lot\_of=time {\F}-go-? {\Inanone}-gather.{\Aor}\\
\trans (1.2) Долго она ходила и собирала. }

\lb{AMKh13}{\gll se-b zaman hege-j zolo t’ulu j-aʁi-dːu\\	 
one-{\Inanone} time {\Dem}-{\F} very strongly {\F}-become\_tired-{\Pf}\\
\trans (1.3) В какой-то момент она очень устала.}

\lb{AMKh14}{\gll hege-j rešu-ʔa hoɢu j-ik’o-dːu\\
{\Dem}-{\F} tree-{\Sup} down {\F}-be-{\Pf}\\
\trans (1.4) Она села на ствол дерева. }

\lb{AMKh15}{\gll se-b zaman b-ik’o-ɬu-kːu b-iči-dːu b-uχːi-dːu berka\\
one-{\Inanone} time {\An}-be-{\Adv}-{\El} {\An}-?-{\Cvb} {\An}-appear-{\Pf} snake\\
\trans (1.5) Как будто из ничего, появилась змея. }

\lb{AMKh16}{\gll ila sir-dːu\\
mother become\_frightened-{\Pf}\\
\trans (1.6) Бабушка испугалась.}

\lb{AMKh17}{\gll hegel-di c’inni<gu>-č’igu čunk’a b-iʁi-dːu berku-ʔa\\
{\Dem}-{\Erg} know<{\Emph}>-{\Neg}.{\Cvb} foot {\Inanone}-stop-{\Pf} snake-{\Sup}\\
\trans (1.7) Она (нечаянно) наступила на змею.}

\lb{AMKh18}{\gll berku-di q’ammi-dːu hegel-ƛi čunk’u-ʔa\\
snake-{\Erg} bite-{\Pf} {\Dem}-{\Gen} foot-{\Sup}\\
\trans (1.8) Змея укусила ее в ногу.}

\lb{AMKh19}{\gll hegel-di b-iχi-dːu hinc’o=lo džab-dːu berku-ʔa\\
{\Dem}-{\Erg} {\Inanone}-take-{\Cvb} stone={\Add} hit-{\Pf} snake-{\Sup}\\
\trans (1.9) Она [бабушка] взяла камень, и бросила [его] на змею.}

\lb{AMKh110}{\gll berka=lo b-ič’o-dːu\\
snake={\Add} {\An}-die-{\Pf}\\
\trans (1.10) Та [змея] умерла.}
 
\lb{AMKh21}{\gll čom-bolo rešin sedu den=no wocːi=lo w-oʔon reš-ƛi he-ɬu-kːu b-ak’arun c’obol.č’at’i\\
some-{\Indef} year ago {\First}{\Sg}={\Add} brother={\Add} {\M}-{\Pl}.go.{\Aor} forest-{\Inter} {\Dem}-{\Adv}-{\El} {\Inanone}-gather.{\Aor} wild\_raspberry\\
\trans (2.1) Несколько лет назад мы с братом шли по лесу и собирали ягоды.}

\lb{AMKh22}{\gll išːil b-ihu=rij w-eƛi b-ak’arun\\
{\First}.{\Excl} {\Inanone}-a\_lot\_of=time {\M}-go.{\Aor} {\Inanone}-gather.{\Aor}\\
\trans (2.2) Долго мы ходили и собирали.}

\lb{AMKh23}{\gll išːil se-b zamana w-aʁi zolo t’ulu\\
{\First}.{\Excl} one-{\Inanone} time {\M}-become\_tired.{\Aor} very strongly\\
\trans (2.3) В какой-то момент мы очень устали. }

\lb{AMKh24}{\gll išːil hoɢu w-ok’o rešu-ƛi hiƛ’u čaχa-ʔa hiʔa\\
{\First}.{\Excl} down {\M}-{\Pl}.be.{\Aor} forest-{\Inter} under ?-{\Sup} ?\\
\trans (2.4) Мы сели на землю под деревом. }

\lb{AMKh25}{\gll se-b zamana b-uχːi berka\\
one-{\Inanone} time {\Inanone}-appear.{\Aor} snake\\
\trans (2.5) Как будто из ничего, появилась змея.}

\lb{AMKh26}{\gll išːil siri\\
{\First}.{\Excl} become\_frightened.{\Aor}\\
\trans (2.6) Мы испугались.}

\lb{AMKh27}{\gll wocːu-di c’inni-č’igu čunk’a b-iʁi\\
brother-{\Erg} know-{\Neg}.{\Cvb} foot {\Inanone}-stop.{\Aor}\\
\trans (2.7) Брат нечаянно наступил на нее.}

\lb{AMKh28}{\gll wocːu-ʔa q’ammi čunk’u-ʔa\\
brother-{\Sup} bite.{\Aor} foot-{\Sup}\\
\trans (2.8) Змея укусила его в ногу. }

\lb{AMKh29}{\gll hegeš-di b-iχi hinc’o=lo džabi berku-ʔa\\
{\Dem}-{\Erg} {\Inanone}-take.{\Aor} stone={\Add} hit.{\Aor} snake-{\Sup}\\
\trans (2.9) Он взял камень, и бросил его на змею. }

\lb{AMKh210}{\gll berka b-ič’o\\
snake {\An}-die.{\Aor}\\
\trans (2.10) Змея умерла.}



\subsection{Kh - Зило} 

\lb{Kh11}{\gll di-qi bosːon di-j ila reš-ƛi j-iʔonni-j ʁurʁi r-ak’arunni-j j-iʔo-j=ʁodi\\
{\First}{\Sg}-{\Instr} tell.{\Aor} {\First}{\Sg}-{\F}[{\Gen}] forest-{\Inter}{\F}-go-{\Cvb} blackcurrant {\Inantwo}-gather-{\Cvb} {\F}-come-{\Pf}={\Rep}\\
\trans (1.1) Мне рассказывали, что однажды моя бабушка шла по лесу, собирала ягоды.}

\lb{Kh12}{\gll χʷant’ulo=lo j-ik’o-j r-ak’arunni-j\\
whole\_day={\Add} {\F}-be-{\Pf} {\Inantwo}-gather-{\Cvb} \\
\trans (1.2) Долго она ходила и собирала. }

\lb{Kh13}{\gll zolo j-aʁi-j j-ik’o-j hege-j\\
very {\F}-become\_tired-{\Cvb} {\F}-be-{\Pf} {\Dem}-{\F}\\
\trans (1.3) В какой-то момент она очень устала.}

\lb{Kh14}{\gll hege-j ho<j>k’o-j rešu-ƛi ʁaduk’olu-ʔa\\
{\Dem}-{\F} sit\_down<{\F}>-{\Pf} forest-{\Inter} treetrunk-{\Sup}\\
\trans (1.4) Она села на ствол дерева. }

\lb{Kh15}{\gll inu-kːu-bolo c’inni-č’igu berka b-uχːi-j herbi\\
where-{\El}-{\Indef} know-{\Neg}.{\Cvb} snake {\An}-appear-{\Pf} then\\
\trans (1.5) Как будто из ничего, появилась змея. }

\lb{Kh16}{\gll ila siri-j\\
mother become\_frightened-{\Pf}\\
\trans (1.6) Бабушка испугалась.}

\lb{Kh17}{\gll hegel-d c’inni-č’igu čunk’a b-iʁi-j berku-ʔa\\
{\Dem}-{\Erg} know-{\Neg}.{Cvb} foot {\Inanone}-stop-{\Pf} snake-{\Sup}\\
\trans (1.7) Она (нечаянно) наступила на змею.}

\lb{Kh18}{\gll berku-di hegel-ƛi čunk’a q’ammi-j\\
snake-{\Erg} {\Dem}-{\Gen} foot bite-{\Pf}\\
\trans (1.8) Змея укусила ее в ногу.}

\lb{Kh19}{\gll hegel-di hinc’o džabi-j berku-ʔa\\
{\Dem}-{\Erg} stone hit-{\Pf} snake-{\Sup}\\
\trans (1.9) Она [бабушка] взяла камень, и бросила [его] на змею.}

\lb{Kh110}{\gll berka b-ič’o-j\\
snake {\An}-die-{\Pf}\\
\trans (1.10) Та [змея] умерла.}
 
\lb{Kh21}{\gll čom=lo rešin sedu wocːi=lo den=no reš-ƛi w-oʔon č’at’i r-ak’arun-nu\\
some={\Add} year ago brother={\Add} {\First}{\Sg} forest-{\Inter} {\M}-{\Pl}.go.{\Aor} blackcurrant {\Inantwo}-gather-{\Inf}\\
\trans (2.1) Несколько лет назад мы с братом шли по лесу и собирали ягоды.}

\lb{Kh22}{\gll zolo b-ihu=ri w-oʁi išil b-ak’aru-mado\\
very {\Inanone}-a\_lot\_of=time {\M}-{\Pl}.be.{\Aor} {\First}.{\Excl} {\Inanone}-gather-{\Prog}\\
\trans (2.2) Долго мы ходили и собирали.}

\lb{Kh23}{\gll se-b zaman išːil zolo w-aʁi\\
one-{\Inanone} time {\First}.{\Excl} very {\M}-become\_tired.{\Aor}\\
\trans (2.3) В какой-то момент мы очень устали. }

\lb{Kh24}{\gll rešu-ƛ’i hiƛ’u ho<wo>k’o išːil\\
tree-{\Sub} under sit\_down<{\M}.{\Pl}>.{\Aor} {\First}.{\Excl}\\
\trans (2.4) Мы сели на землю под деревом. }

\lb{Kh25}{\gll se-b zaman-ʔa b-uχːi berka \\
one-{\Inanone} time-{\Sup} {\An}-appear.{\Aor} snake\\
\trans (2.5) Как будто из ничего, появилась змея.}

\lb{Kh26}{\gll išːil herbi zolo siri\\
{\First}.{\Excl} then very become\_frightened.{\Aor}\\
\trans (2.6) Мы испугались.}

\lb{Kh27}{\gll wocːu-di c’inni-č’igu čunk’a b-iʁi-j hegel-ʔa\\
brother-{\Erg} know-{\Neg}.{\Cvb} foot {\Inanone}-stop-{\Pf} {\Dem}-{\Sup}\\
\trans (2.7) Брат нечаянно наступил на нее.}

\lb{Kh28}{\gll berko-di hegešːu-b čunk’u-ʔa q’ammi\\
snake-{\Erg} {\Dem}-{\Inanone}[{\Gen}] foot-{\Sup} bite.{\Aor} \\
\trans (2.8) Змея укусила его в ногу. }

\lb{Kh29}{\gll hegešː-di b-iχi-j hinc’o džabi berku-ʔa\\
{\Dem}-{\Erg} {\Inanone}-take-{\Cvb} stone hit.{\Aor} snake-{\Sup}\\
\trans (2.9) Он взял камень, и бросил его на змею. }

\lb{Kh210}{\gll berka b-ič’o\\
snake {\An}-die.{\Aor}\\
\trans (2.10) Змея умерла.}




\subsection{KhMM - Зило}

\lb{KhMM11}{\gll di-qi bosːonni-j b-iʁi inna-bolo išːi-j j-oχor-di reš-ƛi j-iʔonni-rbihi r-ak’aru-malo r-ik’o-j č’at’i\\
{\First}{\Sg}-{\Instr} tell-{\Pf} {\Inanone}-be.{\Aor} where-{\Indef} {\First}.{\Excl}-{\F}[{\Gen}] {\F}-old-{\Erg} forest-{\Inter} {\F}-go-{\Temp} {\Inantwo}-gather-{\Prog} {\Inantwo}-be-{\Pf} blackberry\\
\trans (1.1) Мне рассказывали, что однажды моя бабушка шла по лесу, собирала ягоды.}

\lb{KhMM12}{\gll b-ihu=rihi j-ik’o-j hege-j hege-r r-ak’aru-malo\\
{\Inanone}-a\_lot\_of=time {\F}-be-{\Pf}{\Dem}-{\F} {\Dem}-{\Inantwo} {\Inantwo}-gather-{\Prog}\\
\trans (1.2) Долго она ходила и собирала. }

\lb{KhMM13}{\gll se-b zaman hege-j zolo t’ulu j-aʁi-j\\
one-{\Inanone} time {\Dem}-{\F} very strongly {\F}-become\_tired-{\Pf}\\
\trans (1.3) В какой-то момент она очень устала.}

\lb{KhMM14}{\gll hege-j ho<j>k’o-j ruq’ilu-ʔa\\
{\Dem}-{\F} sit\_down<{\F}>-{\Pf} treetrunk-{\Sup}\\
\trans (1.4) Она села на ствол дерева. }

\lb{KhMM15}{\gll c’inni~inni-č’igu b-uχːi-j berka\\
know-{\Neg}.{\Cvb} {\An}-appear-{\Pf} snake\\
\trans (1.5) Как будто из ничего, появилась змея. }

\lb{KhMM16}{\gll ila siri-j\\
mother become\_frightened-{\Pf}\\
\trans (1.6) Бабушка испугалась.}

\lb{KhMM17}{\gll hegel-di c’inni<gu>-č’igu b-iʁi-j berku-ʔa čunk’a\\
\\
\trans (1.7) Она (нечаянно) наступила на змею.}

\lb{KhMM18}{\gll berku-di hegel-ƛi čunk’u-ʔa qammi-j\\
{\Dem}-{\Erg} {\Dem}-{\Gen} foot-{\Sup} bite-{\Pf}\\
\trans (1.8) Змея укусила ее в ногу.}

\lb{KhMM19}{\gll hegel-di b-iχi-j hinc’o=lodu berku-ʔo šammi-j\\
{\Dem}-{\Erg} {\Inanone}-take-{\Cvb} stone={\Sbr} snake-{\Sup}.{\Lat} throw-{\Pf}\\
\trans (1.9) Она [бабушка] взяла камень, и бросила [его] на змею.}

\lb{KhMM110}{\gll herbihi berka b-ič’o-j \\
then snake {\An}-die-{\Pf}\\
\trans (1.10) Та [змея] умерла.}
 
\lb{KhMM21}{\gll čom=lo rešin sedu den=no wocːi=lo w-oʁi reš-ƛi č’at’i r-ak’aru-malo\\
some={\Add} year ago {\First}={\Add} brother={\Add} {\M}-{\Pl}.be.{\Aor} forest-{\Inter} blackberry {\Inantwo}-gather-{\Prog}\\
\trans (2.1) Несколько лет назад мы с братом шли по лесу и собирали ягоды.}

\lb{KhMM22}{\gll b-ihu=rihi w-oʁi išːil he-r r-ak’aru-malo\\
{\Inanone}-a\_lot\_of=time {\M}-{\Pl}.be.{\Aor} {\First}.{\Excl} {\Dem}-{\Inantwo} {\Inantwo}-gather-{\Prog}\\
\trans (2.2) Долго мы ходили и собирали.}

\lb{KhMM23}{\gll se-b zaman išːil t’ulu w-aʁi\\
one-{\Inanone} time {\First}.{\Excl} very {\M}-become\_tired.{\Aor}\\
\trans (2.3) В какой-то момент мы очень устали. }

\lb{KhMM24}{\gll išːil ho<wo>k’o rešu-ƛ’i onši-ʔa hiʔa\\
{\First}.{\Excl} sit\_down<{\M}.{\Pl}>.{\Aor} forest-{\Inter} earth-{\Sup} above\\
\trans (2.4) Мы сели на землю под деревом. }

\lb{KhMM25}{\gll iƛ’i-bolo b-uχːi-j berka\\
?-{\Indef} {\An}-appear-{\Pf} snake\\
\trans (2.5) Как будто из ничего, появилась змея.}

\lb{KhMM26}{\gll išːil siri\\
{\First}.{\Excl} become\_frightened.{\Aor}\\
\trans (2.6) Мы испугались.}

\lb{KhMM27}{\gll wocːu-di c’inni<gu>-č’igu b-iʁi čunk’a berku-ʔa\\
brother-{\Erg} know<{\Emph}>-{\Neg}.{\Cvb} {\Inanone}-stop.{\Aor} foot snake-{\Sup}\\
\trans (2.7) Брат нечаянно наступил на нее.}

\lb{KhMM28}{\gll berku-di čunk’u-ʔa q’ammi-j\\
snake-{\Erg} foot-{\Sup} bite-{\Pf}\\
\trans (2.8) Змея укусила его в ногу. }

\lb{KhMM29}{\gll hegeš-di hinc’o=lo b-iχi-j šammi he-b berku-ʔo\\
{\Dem}-{\Erg} stone={\Add} {\Inanone}-take-{\Cvb} throw.{\Aor} {\Dem}-{\Inanone} snake-{\Sup}.{\Lat}\\
\trans (2.9) Он взял камень, и бросил его на змею. }

\lb{KhMM210}{\gll herbihi berka b-ič’o-(j)\\
then snake {\An}-die.{\Aor}-({\Pf})\\
\trans (2.10) Змея умерла.}


\subsection{M – Зило}

\lb{M11}{\gll di-qi bosːo-mado b-iʁi išːi-j ilu-di reš-ƛi=lo j-iʔonni-j, hel-di biqi b-ak’arunni-j\\
{\First}{\Sg}-{\Instr} tell-{\Prog} {\Inanone}-be.{\Aor} {\First}.{\Excl}-{\F}[{\Gen}] mother-{\Erg} forest-{\Inter}={\Add} {\F}-go-{\Pf} {\Dem}-{\Erg} fruit {\Inanone}-gather-{\Pf}\\
\trans (1.1) Мне рассказывали, что однажды моя бабушка шла по лесу, собирала ягоды.}

\lb{M12}{\gll b-ihu=rihi j-eƛi b-ak’arunni-j\\
{\Inanone}-a\_lot\_of=time {\F}-walk.{\Aor} {\Inanone}-gather-{\Pf}\\
\trans (1.2) Долго она ходила и собирала. }

\lb{M13}{\gll se-b zaman hege-j j-aʁi-j\\
one-{\Inanone} time {\Dem}-{\F} {\F}-become\_tired-{\Pf}\\
\trans (1.3) В какой-то момент она очень устала.}

\lb{M14}{\gll r-ukːu-b rešu-ʔa hiʔa ho<j>k’o-j\\
{\Inantwo}-fall-{\Pst}.{\Ptcp} tree-{\Sup} on sit\_down<{\F}>-{\Pf}\\
\trans (1.4) Она села на ствол дерева. }

\lb{M15}{\gll sebgulo sːu-b-ɬu-kːu b-uχːi-j b-iʔo-j berka\\
noting {\Neg}.{\Cop}-{\Adv}-{\El} {\An}-appear-{\Pf} {\An}-come-{\Pf} snake\\
\trans (1.5) Как будто из ничего, появилась змея. }

\lb{M16}{\gll ila siri-j\\
mother become\_frightened-{\Pf}\\
\trans (1.6) Бабушка испугалась.}

\lb{M17}{\gll c’inni-č’igu berku-ʔa čunk’a b-iʁi-j\\
know-{\Neg}.{\Cvb} snake-{\Sup} foot {\Inanone}-stop-{\Pf}\\
\trans (1.7) Она (нечаянно) наступила на змею.}

\lb{M18}{\gll onšilo berku-di hegel-ƛi čunk’u-ʔa q’ammi-j\\
then snake-{\Erg} {\Dem}-{\Gen} foot-{\Sup} bite-{\Pf}\\
\trans (1.8) Змея укусила ее в ногу.}

\lb{M19}{\gll ilu-di b-iχi-j hinc’o šammi-j berku-ʔo\\
mother-{\Erg} {\Inanone}-take-{\Cvb} stone throw-{\Pf} snake-{\Sup}.{\Lat}\\
\trans (1.9) Она [бабушка] взяла камень, и бросила [его] на змею.}

\lb{M110}{\gll onšilo berka b-ič’o-j\\
then snake {\An}-die-{\Pf}\\
\trans (1.10) Та [змея] умерла.}
 
\lb{M21}{\gll čom=lo rešin sedu den=no wocːi=lo w-oʁi w-oʔinn-e reš-ƛi-kːu\\
some={\Add} year ago {\First}{\Sg}={\Add} brother={\Add} {\M}-{\Pl}.be.{\Aor} {\M}-{\Pl}.go-{\Hab} forest-{\Inter}-{\El}\\
\trans (2.1) Несколько лет назад мы с братом шли по лесу и собирали ягоды.}

\lb{M22}{\gll w-uhol w-eƛi-j, b-ak’arun\\
{\M}-a\_lot\_of.{\Pl} {\M}-walk-{\Cvb} {\Inanone}-gather.{\Aor}\\
\trans (2.2) Долго мы ходили и собирали.}

\lb{M23}{\gll se-b zaman išːil zolo t’ulol w-aʁi\\
one-{\Inanone} time {\First}.{\Excl} very strongly.{\Pl} {\M}-become\_tired.{\Aor}\\
\trans (2.3) В какой-то момент мы очень устали. }

\lb{M24}{\gll onšilo išːil ho<wo>k’o onši-ʔa hiʔa rešu-ƛ’i hiƛ’u\\
then {\First}.{\Excl} sit\_down<{\M}.{\Pl}>.{\Aor} earth-{\Sup} above forest-{\Inter} under\\
\trans (2.4) Мы сели на землю под деревом. }

\lb{M25}{\gll sebgulo.sːu-ɬu-kːu b-uχːi b-iʔo berka\\
nothing.{\Neg}.{\Cop}-{\Adv}-{\El} {\An}-appear.{\Aor} {\An}-come.{\Aor} snake\\
\trans (2.5) Как будто из ничего, появилась змея.}

\lb{M26}{\gll išːil siri\\
{\First}.{\Excl} become\_frightened.{\Aor}\\
\trans (2.6) Мы испугались.}

\lb{M27}{\gll wocːu-č’u-kːu c’inni-č’igu čunk’a b-iʁi hiʔa\\
brother-{\Cont}-{\El} know-{\Neg}.{\Cvb} foot {\Inanone}-stop.{\Aor} above\\
\trans (2.7) Брат нечаянно наступил на нее.}

\lb{M28}{\gll onšilo berku-di hegešu-b čunk’a q’ammi\\
then snake-{\Erg} {\Dem}-{\Inanone}[{\Gen}] foot bite.{\Aor}\\
\trans (2.8) Змея укусила его в ногу. }

\lb{M29}{\gll onšilo hegeš-di berku-ʔa hinc’o džabi\\
then {\Dem}-{\Erg} snake-{\Sup} stone hit.{\Aor}\\
\trans (2.9) Он взял камень, и бросил его на змею. }

\lb{M210}{\gll onšilo berka b-ič’o\\
then snake {\An}-die.{\Aor}\\
\trans (2.10) Змея умерла.}



\subsection{MKG - Зило}

\lb{MKG11}{\gll di-qi bosːon j-oχor ila reš-ƛi=lo j-iʔonni-j ʁʷarq’i r-ak’aru-malo j-iʁi\\
{\First}{\Sg} tell.{\Aor} {\F}-old mother forest-{\Inter}={\Add} {\F}-go-{\Cvb} raspberry {\Inantwo}-gather-{\Prog} {\F}-be.{\Aor}\\
\trans (1.1) Мне рассказывали, что однажды моя бабушка шла по лесу, собирала ягоды.}

\lb{MKG12}{\gll hege-j zolo j-ihu=gu j-eƛi-j, ʁʷarq’i r-ak’arun\\
{\Dem}-{\F} very {\F}-a\_lot\_of={\Emph} {\F}-walk-{\Cvb} raspberry {\Inantwo}-gather.{\Aor}\\
\trans (1.2) Долго она ходила и собирала. }

\lb{MKG13}{\gll se-b zaman here-j j-aʁi-j\\
one-{\Inanone} time {\Dem}-{\F} {\F}-become\_tired-{\Pf}\\
\trans (1.3) В какой-то момент она очень устала.}

\lb{MKG14}{\gll hege-j rešu-ʔa ho<j>k’o\\
{\Dem}-{\F} tree-{\Sup} sit\_down<{\F}>.{\Aor}\\
\trans (1.4) Она села на ствол дерева. }

\lb{MKG15}{\gll inu-kːu=gu b-ik’o-bolo b-uχːi-j b-iʔo berka\\
where-{\El}={\Emph} {\Inanone}-be-{\Indef} {\An}-appear-{\Cvb} {\An}-come.{\Aor} snake\\
\trans (1.5) Как будто из ничего, появилась змея.}

\lb{MKG16}{\gll ila siri berku-č’u-kːu\\
mother become\_frightened.{\Aor} snake-{\Cont}-{\El}\\
\trans (1.6) Бабушка испугалась.}

\lb{MKG17}{\gll hegel-di j-iχi-j čunk’a b-iʁi-j berku-ʔa\\
{\Dem}-{\Erg} {\F}-?-{\Cvb} foot {\Inanone}-stop-{\Pf} snake-{\Sup}\\
\trans (1.7) Она (нечаянно) наступила на змею.}

\lb{MKG18}{\gll berku-di ilu-ƛi čunk’u-ʔa q’ammi\\
snake-{\Erg} mother-{\Gen} foot bite.{\Aor}\\
\trans (1.8) Змея укусила ее в ногу.}

\lb{MKG19}{\gll ilo-di b-iχi-j hinc’o šammi berku-ʔo\\
mother-{\Erg} {\Inanone}-take-{\Cvb} stone throw.{\Aor} snake-{\Sup}.{\Lat}\\
\trans (1.9) Она [бабушка] взяла камень, и бросила [его] на змею.}

\lb{MKG110}{\gll berka b-ič’o\\
snake {\An}-die.{\Aor}\\
\trans (1.10) Та [змея] умерла.}
 
\lb{MKG21}{\gll čom=lo rešin sedu wocːi=lo den=no ʁʷarq’i r-ak’aru-malo w-oʁi reš-ƛi\\
some={\Add} year ago brother={\Add} {\First}{\Sg}={\Add} raspberry {\Inantwo}-gather-{\Prog} {\M}-{\Pl}.be.{\Aor} forest-{\Inter}\\
\trans (2.1) Несколько лет назад мы с братом шли по лесу и собирали ягоды.}

\lb{MKG22}{\gll išːil zolo w-uhol w-eƛi-j r-ak’arun ʁʷarq’i\\
{\First}.{\Excl} very {\M}-a\_lot\_of.{\Pl} {\M}-walk-{\Cvb} {\Inantwo}-gather.{\Aor} raspberry\\
\trans (2.2) Долго мы ходили и собирали.}

\lb{MKG23}{\gll išːil zolo w-aʁi\\
{\First}.{\Excl} very {\M}-become\_tired.{\Aor}\\
\trans (2.3) В какой-то момент мы очень устали. }

\lb{MKG24}{\gll išːil hoʔor-ʔa ho<wo>k’o rešu-ƛ’i\\
{\First}.{\Excl} ground-{\Sup} sit\_down<{\M}.{\Pl}>.{\Aor} forest-{\Inter}\\
\trans (2.4) Мы сели на землю под деревом. }

\lb{MKG25}{\gll inu-kːu=gu b-ik’o-bolo b-uχːi berka\\
where-{\El}={\Emph} {\Inanone}-be-{\Indef} {\An}-appear.{\Aor} snake\\
\trans (2.5) Как будто из ничего, появилась змея.}

\lb{MKG26}{\gll išːil siri\\
{\First}.{\Excl} become\_frightened.{\Aor}\\
\trans (2.6) Мы испугались.}

\lb{MKG27}{\gll wocːu-di čunk’a b-iʁi berku-ʔa\\
brother-{\Erg} foot {\Inanone}-stop.{\Aor} snake-{\Sup}\\
\trans (2.7) Брат нечаянно наступил на нее.}

\lb{MKG28}{\gll berku-di wocːu-ʔa q’ammi\\
snake-{\Erg} brother-{\Sup} bite.{\Aor}\\
\trans (2.8) Змея укусила его в ногу. }

\lb{MKG29}{\gll hegeš-di b-iχi-j hinc’o šammi berku-ʔo\\
{\Dem}-{\Erg} {\Inanone}-take-{\Cvb} stone throw.{\Aor} snake-{\Sup}.{\Lat}\\
\trans (2.9) Он взял камень, и бросил его на змею. }

\lb{MKG210}{\gll berka b-ič’o\\
snake {\An}-die.{\Aor}\\
\trans (2.10) Змея умерла.}



\subsection{Z - Зило}

\lb{Z11}{\gll di-qi bosːon se-b onši di-j ila reš-ƛi=lo j-iʔonni-j ʁʷarq’i r-ak’arunni-j\\
{\First}{\Sg}-{\Instr} tell.{\Aor} one-{\Inanone} time {\First}{\Sg}-{\F}[{\Gen}] mother forest-{\Inter}={\Add} {\F}-go-{\Pf} blackcurrant {\Inantwo}-gather-{\Pf}\\
\trans (1.1) Мне рассказывали, что однажды моя бабушка шла по лесу, собирала ягоды.}

\lb{Z12}{\gll b-ihu=ri=lo j-eƛi-j r-ak’arunni-j\\
{\Inanone}-a\_lot\_of=time={\Add} {\F}-walk-{\Cvb} {\Inantwo}-gather-{\Pf} \\
\trans (1.2) Долго она ходила и собирала.}

\lb{Z13}{\gll hege-b zaman zolo j-aʁi-j hege-j\\
{\Dem}-{\Inanone} time very {\F}-become\_tired-{\Pf} {\Dem}-{\F}\\
\trans (1.3) В какой-то момент она очень устала.}

\lb{Z14}{\gll hege-j ho<j>k’o-j ʁaduk’ollo-ʔa\\
{\Dem}-{\F} sit\_down<{\F}>-{\Pf} treetrunk-{\Sup}\\
\trans (1.4) Она села на ствол дерева. }

\lb{Z15}{\gll inu-kːu-bolo c’inni-č’igu berka b-uχːi-j\\
where-{\El}-{\Indef} know-{\Neg}.{\Cvb} snake {\An}-appear-{\Pf}\\
\trans (1.5) Как будто из ничего, появилась змея. }

\lb{Z16}{\gll ila siri\\
mother become\_frightened.{\Aor}\\
\trans (1.6) Бабушка испугалась.}

\lb{Z17}{\gll hegel-di c’inni-č’igu b-iʁi-j berku-ʔa\\
{\Dem}-{\Erg} know-{\Neg}.{\Cvb} {\Inanone}-stop-{\Pf} snake-{\Sup}\\
\trans (1.7) Она (нечаянно) наступила на змею.}

\lb{Z18}{\gll hegel-di b-iχi-j šammi-j hinc’o berku-ʔo\\
{\Dem}-{\Erg} {\Inanone}-take-{\Pf} throw-{\Pf} stone snake-{\Sup}.{\Lat}\\
\trans (1.8) Змея укусила ее в ногу.}

\lb{Z19}{\gll berku-di hegel-ƛi čunk’u-ʔa q’ammi-j\\
snake-{\Erg} {\Dem}-{\Gen} foot-{\Sup} bite-{\Pf}\\
\trans (1.9) Она [бабушка] взяла камень, и бросила [его] на змею.}

\lb{Z110}{\gll berka b-ič’o\\
snake {\An}-die.{\Aor}\\
\trans (1.10) Та [змея] умерла.}
 
\lb{Z21}{\gll čom=lo rešin sedu den=no wocːi=lo w-oʔonni-j reš-ƛi=lodu č’at’i r-ak’arun\\
some={\Add} year ago {\First}{\Sg}={\Add} brother={\Add} {\M}-{\Pl}.go-{\Cvb} forest-{\Inter}={\Sbr} blackcurrant {\Inantwo}-gather.{\Aor}\\
\trans (2.1) Несколько лет назад мы с братом шли по лесу и собирали ягоды.}

\lb{Z22}{\gll b-ihu=ri w-eƛi-j r-ak’arun\\
{\Inanone}-a\_lot\_of=time {\M}-walk-{\Cvb} {\Inantwo}-gather.{\Aor}\\
\trans (2.2) Долго мы ходили и собирали.}

\lb{Z23}{\gll hege-b zaman išːil zolo w-aʁi\\
{\Dem}-{\Inanone} time {\First}.{\Excl} very {\M}-become\_tired.{\Aor}\\
\trans (2.3) В какой-то момент мы очень устали. }

\lb{Z24}{\gll išːil ho<wo>k’o hoʔor-ʔa rešu-ƛ’il\\
{\First}{\Sg} sit\_down<{\M}.{\Pl}> ground-{\Sup} tree-{\Sub}\\
\trans (2.4) Мы сели на землю под деревом. }

\lb{Z25}{\gll se-b b-ik’o-b-ɬu-kːu b-ič’i-j b-uχːi berka\\
one-{\Inanone} {\Inanone}-be-{\Pst}.{\Ptcp}-{\Adv}-{\El} {\An}-?-{\Cvb} {\Inanone}-appear.{\Aor} snake\\
\trans (2.5) Как будто из ничего, появилась змея.}

\lb{Z26}{\gll išːil siri\\
{\First}.{\Excl} become\_frightened.{\Aor}\\
\trans (2.6) Мы испугались.}

\lb{Z27}{\gll wocːu-di c’inni<gu>-č’igu b-iʁi-j hegel-ʔa\\
brother-{\Erg} know<{\Emph}>-{\Neg}.{\Cvb} {\Inanone}-stop-{\Pf} {\Dem}-{\Sup}\\
\trans (2.7) Брат нечаянно наступил на нее.}

\lb{Z28}{\gll berku-di q’ammi-j hegešːu-b čunk’u-ʔa\\
snake-{\Erg} bite-{\Pf} {\Dem}-{\Inanone}[{\Gen}] foot-{\Sup}\\
\trans (2.8) Змея укусила его в ногу. }

\lb{Z29}{\gll hegeš-di b-iχi-j hinc’o šammi-j hegel-ʔo\\
{\Dem}-{\Erg} {\Inanone}-take-{\Cvb} stone throw-{\Pf} {\Dem}-{\Sup}.{\Lat}\\
\trans (2.9) Он взял камень, и бросил его на змею. }

\lb{Z210}{\gll berka b-ič’o\\
snake {\An}-die.{\Aor}\\
\trans (2.10) Змея умерла.}

\end{appendices}
