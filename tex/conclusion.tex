
\chapter*{Заключение}
\addcontentsline{toc}{chapter}{Заключение}

Цель настоящей диссертации --- охарактеризовать грамматический статус перфектоподобных форм в нахско-дагестанских языках и, в особенности, то место, которое занимает в значении этих форм компонент эвиденциальности, а также оценить гипотезу о том, что эвиденциальное использование этих форм мотивировано контактами с местными тюркоязычными народами. Нахско-дагестанские языки находятся в центре большого ареала, где встречаются сходные системы выражения эвиденциальности (в основном: косвенной засвидетельствованности) с разным уровнем грамматикализации. Вопрос о том, что является, а что не является грамматическим способом выражения эвиденциальности, занимает в типологических исследованиях этой категории центральное место, так как во многих языках мира обнаруживаются формы, находящиеся на той или иной стадии грамматической эволюции в направлении эвиденциальных значений, и формы, значение которых имеет более или менее выраженные эвиденциальные "оттенки". В первой главе мы предложили различать недограмматикализованные конструкции, которые приобретают эвиденциальное прочтение только в определенном дискурсивном контексте, и те, у которых интерпретация (частично) определяется морфосинтаксисом. Так, например, в  некоторых нахско-дагестанских языках, доступные интерпретации многозначной формы перфекта могут зависеть в том числе от вида глагола --- имперфективные глаголы в аштынском даргинском по умолчанию получают эвиденциальное прочтение, см. обсуждение в \ref{sec:cat}.
\par Эвиденциальность в целом можно рассматривать как дейктическую (в другой терминологии --- индексальную) категорию, поскольку она определенным образом характеризует связь между неким событием и речевым актом. Дейктические подходы удобны тем, что они позволяют анализировать употребление эвиденциальных показателей по аналогии с другими дейктическими (индексальными) категориями, например, по аналогии с категорией времени, а также с категорией эпистемической модальности. При этом учитываются различные компоненты события, разные роли говорящего, их отождествление и <<растождествление>>, а также соответствующие сдвиги эпистемической перспективы, то есть то, как говорящий манипулирует этой перспективой, либо приближая действие к моменту речи (например через употребление praesens historicum или формы личной засвидетельствованности), либо, наоборот, дистанцируясь от него. Кроме того, интерпретация эвиденциальности как дейктической категории позволяет описывать данную семантическую зону как континуум, а не как таксономию значений и оппозиций. 
%здесь в принципе продолжает оставаться неясным, почему следующее вытекает/становится возможным именно в дейктических подходах. по крайней мере для меня
Деление на прямую и непрямую информацию, например, релевантно не для всех языков, поскольку значение инферентива может сочетаться в одном показателе с репортативом (косвенная засвидетельствованность) или со значением прямого восприятия (прямая, или личная засвидетельствованность). Дейктическая интерпретация позволяет также исключить квотатив из семантической зоны эвиденциальности. Хотя квотативные показатели могут находиться в одной парадигме с эвиденциальными показателями (показатели эвиденциальности вообще нередко образуют единую парадигму с показателями других категорий),  они указывают на источник информации только косвенно. Их основная функция --- указать на то, что высказывание является цитатой (часто они маркируют ее границы, подобно кавычкам в письменном тексте); в некоторых случаязх - ввести говорящего - источник цитаты. Квотативы могут иногда выполнять функцию репортатива --- такое употребление отмечено и у некоторых нахско-дагестанских частиц передачи чужой речи --- но в конкретных контекстах квотативная и репортативная функции чаще всего могут быть различены. Особое место в типологии эвиденциальности занимает отсылка к общему знанию --- знанию, которое человек осваивал в течение жизни из разных конкретных источников и которое он затрудняется приписать какому-то конкретному, <<дискретному>> источнику.
\par Во второй главе мы определили термин <<перфектоид>> как обозначающий форму, которая имеет значение результатива или текущей релевантности, а также, возможно, значение косвенной засвидетельствованности (\ref{sec:perf}). Среди них бывают такие формы, которые более похожи на категорию перфекта как она известна в типологии, и такие, которые явно имеют такое же происхождение, но чью функцию сложно называть перфектной. Такого рода формы представлены во всех нахско-дагестанских языках. Они восходят к результативной конструкции, чаще всего представленной аналитической конструкцией на основе (общего перфективного) конверба и копулы настоящего времени. Исходная результативная конструкция семантически достаточно похожа на результативный перфект, но последний считается уже последующим этапом грамматической эволюции результатива. Для типологии нахско-дагестанских перфектоидов важно различать эти категории, так как здесь <<узкие>> результативы иногда представлены отдельными формами, существующими в дополнение к более грамматикализованным перфектам. Как мы показали, для различения этих категорий важным параметром является возможность присутствия действующего субъекта --- узкий результатив не допускает такую возможность и либо отвергается, либо семантически реинтерпретируется носителями. Помимо перфектоидной эвиденциальности в языках семьи представлены и другие способы выражения данной категории, в том числе специализированные частицы для передачи, в частности, репортатива, а также их диахронические предшественники --- клитизированные глаголы речи и специализированные вспомогательные глаголы (например `быть' в форме перфекта, глаголы `найти', `оставаться', и т.д.).
\par Ареалы распространения эвиденциальных частиц и вспомогательных глаголов не позволяют выявить яркий генеалогический или ареальный паттерн (за исключением конструкции с `найти', ср. \citep{danielmaisak2018}). В случае перфектоидов, развитие определенной формальной структуры также скорее показывает параллельный дрейф для всех языков семьи. С семантикой дело обстоит иначе. Тогда как семантика результатива и текущей релевантности свойственна всем языкам, эвиденциальность как значение перфектоида не характерна для лезгинских языков на юге, но в то же время широко представлена на севере и северо-западе Дагестана. Возможно, это связано со сферой влияния разных тюркских языков. На юге региона традиционно был распространен азербайджанский язык --- тюркский язык огузской группы, в котором эвиденциальная семантика прошедшего времени на \textit{-mIš} выражена слабо\citep{johanson2018}. В центральной и северо-западной зоне исторически роль лингва франка играл кумыкский язык (наряду с аварским) \citep{chirikba2008}, хотя влияние кумыкского языка на языки севера и северо-запада было значительно слабее, чем влияние азербайджанского на юге.
% кажется у Анны Владимировны было замечание, что в кумыкском что-то тоже не очевидно. Но по текстам ведь оказалось похоже на андийский, да? у нее у самой?
\par В третьей главе мы рассмотрели употребление перфектоидов в нарративных текстах. В языках с эвиденциальными использованиями перфектоида эта форма часто выступает в рассказах о незасвидетельствованных событиях, тогда как прототипические перфекты не могут оформлять главную линию нарратива. Было предложено считать, что использование перфектоидов в (заглазном) нарративе может служить объективной сопоставительной мерой степени его грамматикализованности, которая иногда может противоречить результатам анализа, опирающегося на данные типологических анкет. Перфектоиды значительно более частотны в заглазных нарративах, чем в других типах дискурса, а частотность общего прошедшего как нейтрального варианта ниже в случае, когда перфектоид как заглазное прошедшее более грамматикализован. С другой стороны, нарративное употребление перфектоидов отсутствует в языках, где, по данным других исследователей, у них отсутствует сильно грамматикализованная эвиденциальность. 
%если это про типа цахурский, то там же все же тоже есть нарративные употребления в текстах, нет? я бы тогда говорил про частотность, а не прямо "отсутствуют"
Употребление перфектоида в заглазном нарративе представлено и в тех языках или диалектах, где эвиденциальная семантика в других контекстах слабо ощущается (напр. зиловский диалект андийского). В связи с этим встает вопрос, не может ли язык заимствовать только нарративное использование перфектоида как стилистический прием. Наши данные позволяют отвергнуть это предположение --- эвиденциальное ядро необходимо для развития формы в сторону заглазного прошедшего, включая нарративную фукнцию.
\par В третьей главе мы предложили возможный механизм распространения нарративного перфектоида в изучаемом регионе, опираясь на понятия minimally counterintuitive concept (MCI) и perceptual magnet (перцептивный магнит). Идея заключается в том, что эвиденциальный перфект одного языка для носителя другого, контактирующего языка, в котором имеется перфектоид текущей релевантности, является, с одной стороны, интуитивно понятной сущностью (формы обоих языков выражают текущую релевантность), но, с другой стороны, нарушает одну из главных диагностик перфектности --- используется как основная форма в нарративных цепочках. Нарративное употребление тем самым является перцептивно заметной, необычной функцией перфектоида. Это сочетание знакомого и нового превращает нарративное употребление в MCI, концепт, который легко запоминается и заимствуется. Л. Йохансон предполагает, что, в том что касается эвиденциальности, тюркские языки на Кавказе лишь способствовали усилению внутриязыковых тенденций развития \citep{johanson2006}. В связи с этим нам кажется вероятным, что при интенсивном обмене традиционными жанрами именно нарративное употребление перфектоида играло роль перцептивного магнита (см. раздел \ref{sec:itogi3} и \citep{blevins2017}) и стимулировало развитие у перфектоидов эвиденциальной семантики.
%я бы еще раз повторил, что вы все же считаете, что без наличия у перфектоида эвиденцаильной семантики такое развитие было бы невозможно - а то этот абзац входит в противоречие со сказанным выше. Типа: "и стимулировало у перфектоидов развитие эвиденциальной семантики (в том или ином виде уже отчасти присутствующей)." Или же в самом конце, в финальном абзаце, написать типа "По-видимому, в ходе развития перфектоидов в эвиденциальные формы языковые контакты играли не центральную, а вспомогательную роль." - или что-то такое.
%но я продолжаю быть согласным с рецензентами, что эта логика MCI, хоть и красивая и неожиданная, все же пока слабоватая.
\par Подводя итоги обсуждению гипотезы о контактном происхождении эвиденциальности как значения перфектоида, следует признать, что мы не смогли установить конкретный сценарий заимствовании признака, при котором язык А в определенный момент времени заимствует признак из языка Б в документированной ситуации многоязычия. С другой стороны, на Восточном Кавказе можно выделить два ареала, в соответствии с присутствием (центральный / северо-западный ареал) или отсутствием (южный ареал) нарративного перфектоида, что в целом коррелирует с распространением разных тюркских языков в качестве лингва франка: кумыкский, в котором присутствует эвиденциальная семантика, в первом ареале, и азербайджанский, в котором она отсутствует, во втором ареале. Ареальная гипотеза о развитии у перфектоидов эвиденциальной семантики требует дальнейшего изучения, в том числе детального анализа грамматического оформления разного рода нарративов с учетом социолингвистических, антропологических и исторических данных.
